\documentclass{article}
\usepackage[UTF8]{ctex}
\usepackage{geometry}
\usepackage{multirow}
\usepackage{natbib}
\geometry{left=3.18cm,right=3.18cm,top=2.54cm,bottom=2.54cm}
\usepackage{graphicx}
\pagestyle{plain}	
\usepackage{setspace}
\usepackage{enumerate}
\usepackage{caption2}
\usepackage{datetime} %日期
\renewcommand{\today}{\number\year 年 \number\month 月 \number\day 日}
\renewcommand{\captionlabelfont}{\small}
\renewcommand{\captionfont}{\small}
\begin{document}
	
	\begin{figure}
		\centering
		\includegraphics[width=8cm]{upc.png}
		
		\label{figupc}
	\end{figure}
	
	\begin{center}
		\quad \\
		\quad \\
		\heiti \fontsize{45}{17} \quad \quad \quad 
		\vskip 1.5cm
		\heiti \zihao{2} 《计算科学导论》个人职业规划
	\end{center}
	\vskip 2.0cm
	
	\begin{quotation}
		% 	\begin{center}
		\doublespacing
		
		\zihao{4}\par\setlength\parindent{7em}
		\quad 
		
		学生姓名:\underline{\qquad  于金龙 \qquad \qquad}
		
		学\hspace{0.61cm} 号:\underline{\qquad 2007010326\qquad}
		
		专业班级:\underline{\qquad 计科2003 \qquad  }
		
		学\hspace{0.61cm} 院:\underline{计算机科学与技术学院}
		% 	\end{center}
		\vskip 1.5cm
		\centering
		\begin{table}[h]
			\centering 
			\zihao{4}
			\begin{tabular}{|c|c|c|c|c|c|c|c|c|}
				% 这里的rl 与表格对应可以看到,姓名是r,右对齐的;学号是l,左对齐的;若想居中,使用c关键字。
				\hline
				\multicolumn{5}{|c|}{分项评价} &\multicolumn{2}{c|}{整体评价}  & 总    分 & 评 阅 教 师\\
				\hline
				自我 & 环境 & 职业 & 实施 & 评估与 & 完整性 & 可行性 &\multirow{2}*{} &\multirow{2}*{}\\
				分析& 分析& 定位 & 方案 & 调整 & 20\% & 20\% & ~&~ \\\            
				10\% & 10\% & 15\% & 15\% & 10\% & &  &~ &~\\
				\cline{1-7} 
				& & & & & & & ~&~ \\
				& & & & & & & ~&~ \\
				\hline      
			\end{tabular}
		\end{table}
		\vskip 2cm
		\today
	\end{quotation}
	
	\thispagestyle{empty}
	\newpage
	\setcounter{page}{1}
	% 在这之前是封面,在这之后是正文
	\section{自我分析}
	先用了几分钟做了一下比较权威的自我测试:艾森克人格问卷(EP0),个人觉得都比较符合我的性格,测试报告如下:\par
	1.精神质得分:52.881,属于中间型,能较好地适应外部环境,态度温和。\par
	2.性格内外向得分:43.023,属于内向型,除了亲密朋友外,对一般人比较冷淡,不喜冒险,喜欢有规律的生活方式。\par
	3.神经质/情绪稳定性得分:45.909,属于中间型(稳定性),遇到刺激有较为强烈的情绪反应,大多情况下可以自控。\par
	4.掩饰性得分:39.747,倾向诚实型,因此我的问卷应该是符合事实的。\par
	并不是要通过一张问卷来对我进行完全的分析,而是通过这份问卷中的问题和角度,我发现了自己性格中的优点和缺点。\par
	先来说一下优点吧,(虽然有点不好意思)\par
	1.我属于比较内向型的选手,平时遇到难题,更倾向于通过CSDN,洛谷,百度百科等途径来查询题解,不经常和学长老师交流,(我的印象中就问过郭楠学长一次DP的题)。更喜欢一个人钻研一天,就是比较耐得住性子。\par
	2.在和他人交换意见时,我会适当的做出让步,不执着的坚持自己的意见,导致作业或项目始终卡在一个阶段,不能进行下去。(如果我就是对的,那么我是不会放弃的。)\par
	缺点方面:\par
	1.我觉得我的性格可能会对我的职业影响很多,在EPO测试时,我觉得自己并不能很及时的控制住自己的情绪,包括但不限于愤怒,失落,沮丧等,也就是说,如果给我一个无情(脑)的甲方,我可能会更倾向于避而远之,这样很可能会导致我们的意见想法达不到一致,最终不欢而散。\par
	有时仅凭个人喜好或价值观来做某件事,如果被他人恶意引导,大概率第一时间不会察觉,因此在了解某件事之前,我一般都想去知乎看看大佬们的说法(虽然有时知乎上也是众说纷纭)。\par
	
	\section{环境分析}
	
	\subsection{1.家庭环境分析}
	家庭关系和睦,经济情况一般,父母文化水平不高,所以希望我有文化,由于父亲曾经时军人,更希望我子承父业,但我觉得计算机专业更有前途,母亲更理解我的想法。现在更希望的是留在一线城市发展。\par
	\subsection{学校环境分析}
	由于我参加acm后备役,或多或少的受到学长以及老师的熏(xi)陶(nao),每年acm都会有队员进入大厂,当然,我不会侥幸的认为我会是那一个,因为学长们肯定在竞赛上花费了无数的心血才有今天的成就,他们的成功只是告诉我一个道理:坚持到最后,努力至少是有成果的。让我有了一个可以坚信并为之奋斗的榜样。\par
	\subsection{3.社会环境分析}
	很多人说计算机行业饱和,这是对于一些培训机构批量产生计算机行业初级人员而言,大批量制造的结果就是大量掌握同等技术,水平差不多的人进入就业市场产生供应过剩,但是高级别人才,只要行业需求没变,一直都是稀缺的。因此,仅仅是学习课本中的内容是不够的,如果对自己有更高的要求,应该时刻学习新的知识,才能不至于被后浪淘汰。\par
	
	
	\subsection{职业环境分析}
	首先,行业及区域分布方面,全国计算机专业主要集中在计算机软件,新能源,互联网行业。而就业区域主要集中在北上广深四大一线城市,占比高达百分之58。\par
	其次,就业方向方面,全国计算机类本科生毕业期望从业方向主要为系统工程师,软件工程师,硬件工程师,游戏开发/设计,IT项目经理。其中硬/软件工程师,游戏开发/设计成为近几年的热门就业趋势。\par
	最后,薪资水平方面,游戏开发/设计应届生薪酬高达8500/月,其次是软件工程师,迎接薪酬突破6000/月,总体来说,这两个方向是最吃香的两个方向。\par
	
	\section{职业定位}
	从薪资方面来说,游戏开发似乎更为吃香,既能拿高额工资有可以从事自己热爱的工作,可以说是两全其美,但是,目前国内的游戏开发似乎都围绕着如何做出一款赚钱的网游,而不是如何做出一款有趣的单机,凡事总需研究才会明白,以前时常被骗氪,我也还记得,可是不甚清楚。我翻开支付记录一查,这记录没有时间,歪歪斜斜的每页上都写着“氪金变强”几个字。我横竖睡不着,仔细看了半天,才从屏幕里看出字来,满屏幕都写着两个字——“骗氪”。没有骗过氪的程序员,或者还有?救救孩子......情不自禁用鲁迅先生的文章来做了一个不太恰当的类比,但我相信周先生一定是参透了点什么,曾经的“吃人”只不过是以另一种形式寄托在了游戏中。因此,我可能更适合去从使用者的角度去观察游戏开发,并不能与开发者们共情。再者,我相信任何一个有志青年,都不会任由资本对自己的心血肆意蹂躏,这其中或许有着我现在无法体会的压力,迫使他们不得不“吃人”。根据我所做过的EPO测试来看,我显然是没有很高的抗压能力。\par
	身为一名有志向,有理想的青年,不应该把薪资放在第一位,虽然它在一定程度上反映了这个职业所需求的技术难度与社会需求,但适合自己的才是最好的。\par
	我觉得目前最适合我的定位应该是软件工程师.\par
	\subsection{行业领域定位与理由}
	软件工程师,软件工程师属于后端工程师的一种,更接近业务核心,侧重业务功能的实现,处理的大多是性能问题。首先,据我了解,字节跳动招聘现状是:算法工程师过剩,开发工程师稀缺,且开发工程师的工资没有出现下降趋势,这意味着开发工程师人才并未趋向饱和甚至是稀缺,不只是头条,很多互联网公司都有这种情况,因此一个合格的软件工程师的从业竞争压力是比较小的。\par
	\subsection{职业目标与可行性分析}
	我的职业目标就是成为一个合格的软件工程师,合抱之木,生于毫末,九层之台,起于垒土。想要称得上合格二字,至少需要5年以上的工作经验,而且需要亲自参与一些项目,而仅凭我大学四年来说,是远远不够的,因此,我所能做的就是先成为一个初级软件工程师(程序员)。\par
	
	\section{实施方案}
	第一,要学习多种语言。\par
	先从学习C++这一门语言开始,而不是尝试阳光下的一切,否则只会导致混乱,将学习其它技术所需要的精力浪费。在初步掌握C++后可以学习其他语言,最重要的是要将一门语言精通,并了解来龙去脉。\par
	第二,学会如何构造代码。\par
	在学习一门语言后或者同时,学会编写优秀,清晰,可理解的代码,因为代码本身是可以交流的,因此不需要大量注释。其一,结构化代码是软件开发中的艺术部分,是对自己心血的认可,而不仅仅是完成任务。其二,结构化代码能节省维护时间。\par
	第三,熟悉以下必备技术\par
	1.熟悉net开发网站流程,包括ASP等数据库开发;\par
	2.熟悉java开发流程,特别是java流行框架ssh,ssm等;\par
	3.熟悉web开发技术,前端流行开发框架,vuejs等;\par
	4.熟悉php开发,特别是MVC模式\par
	第四,学习并能熟练运用各种算法。\par
	算法是解决各种计算机科学/编程问题的常用方法,通常擅长算法可以是使一个开发人员在一个小时内解决问题,而另一位则需要花费几天才能弄清楚。就我目前所接触到的算法有前缀和,差分,贪心,排序,动态规划,并查集等,它们都是为了减少时间复杂度与空间复杂度所用的工具,但是,除非熟悉并精通这些算法,否则甚至不知道已经存在了一个优雅的解决方案。\par
	第五,第五,学习并能熟练运用各种数据结构。\par
	所有的软件开发人员都应该熟悉几种数据结构,包括:1.数组和向量 2.链表 3.堆栈 4.Queue列 5.树木 6.散列 7.套装 通过很好的掌握数据结构与算法,可以轻松优雅的解决许多困难问题。目前我所接触的数据结构有数组,向量,集合,映射,队列,优先队等,只是停留在认识方面,a.push,a.insert,a.top,a.fiont什么的经常会混乱,因此我觉得学习一种数据结构应该在掌握了一种之后再去学下一个。\par
	
	
	\section{评估与调整}
	根据我所做的实施方案,我觉得首先要做到的是充分利用接下来的寒假,将这学期学到但未做到灵活运用的一些算法和数据结构掌握,而想要掌握这些必然要到洛谷,力扣,CF,我校ACM等刷题网站练习,因此,第一个简单的评估标准就是题量,而为了防止我有水题的嫌疑,我觉得第二个评估标准就是难度,例如洛谷中将贪心问题分为入门,普及,普及/提高,提高/省选,省选/NOI,NOI/NOI+/CTSC,几个层次,我觉得先将水平提高到省选不错了。
	考虑到寒假我们会进行比赛,网课,训练赛等,计划会有所改变,我的调整原则就是追求质量,不追求数量,能掌握几个算法和数据结构就用几个,过分的追求进度,反而会导致成果下降。
	在下个学期的课程里,我们将要学python,根据我所做的实施方案,要掌握多种语言,在接触python时,将C++与之对比,发现python真正的优点与缺点,不少掌握了python的同学都曾说过,python只用几行就写完了,而掌握了C++的同学则说python运行速度非常慢,C++才是最好的语言。李昕老师的解释是这样的:python真正的优势就是能调用C++等,因此不必抓着它本身速度慢的缺点不放。评估标准就是期末考试了吧,毕竟现在还是靠C++吃饭的,对python做到普及程度就好。
	至于成为软件开发工程师必须要熟悉的开发技术和工具,我觉得可以在日常的学习空暇中进行了解,评估的标准则是在走出校园时,掌握了的程度,调整原则则是以课程学习为主,做到学有余力时进行学习。\par 
	\subsection{评估时间}
	每次评估时间为假期后与学期末\par
	\subsection{评估内容}
	对于acm来说,第一个简单的评估标准就是题量,而为了防止我有水题的嫌疑,我觉得第二个评估标准就是难度,例如洛谷中将贪心问题分为入门,普及,普及/提高,提高/省选,省选/NOI,NOI/NOI+/CTSC,几个层次,我觉得先将水平提高到省选不错了。\par
	\subsection{调整原则}
	我的调整原则就是追求质量,不追求数量,能掌握几个算法和数据结构就用几个,过分的追求进度,反而会导致成果下降。\par
	
	
	
	
\end{document}